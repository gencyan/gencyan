\documentclass[11pt, a4paper]{article}
\renewcommand{\baselinestretch}{1.5}
\usepackage[top=1.5cm,bottom=2cm]{geometry}
\usepackage[utf8x]{inputenc} 
\usepackage{ucs}
\usepackage{amsmath}
\usepackage{amsfonts}
\usepackage[T1,T2A]{fontenc}
\usepackage{amssymb}
\usepackage{makeidx}
\usepackage[russian]{babel}

\begin{document}

 \begin{center} 
  {\LARGE Формула Бернули} 
 \end{center}

Формула Бернулли — формула в теории вероятностей, позволяющая находить вероятность появления события \textit{A} при независимых испытаниях. Формула Бернулли позволяет избавиться от большого числа вычислений — сложения и умножения вероятностей — при достаточно большом количестве испытаний. Названа в честь выдающегося швейцарского математика Якоба Бернулли, который вывел эту формулу.

 \begin{center}
   \textbf{Формулировка}
 \end{center}

\textbf{Теорема}. Если вероятность \textit{p} наступления события \textit{A} в каждом испытании постоянна, то вероятность $ P_{k,n}$ того, что событие \textit{A} наступит ровно \textit{k} раз в \textit{n} независимых испытаниях, равна: $ P_{k,n}=C^{k}_{n}\cdot p^{k}\cdot q^{n-k} $, где  $q=1-p$.

 \begin{center}
   \textbf{Доказательство}
 \end{center}
 
 Пусть проводится \textit{n} независимых испытаний, причём известно, что в результате каждого испытания событие \textit{A} наступает с вероятностью $ P\left(A\right)=p $ и, следовательно, не наступает с вероятностью $ P\left({\bar {A}}\right)=1-p=q $. Пусть, так же, в ходе испытаний вероятности \textit{p} и \textit{q} остаются неизменными. Какова вероятность того, что в результате \textit{n} независимых испытаний, событие \textit{A} наступит ровно \textit{k} раз?

Оказывается можно точно подсчитать число "удачных" комбинаций исходов испытаний, для которых событие \textit{A} наступает \textit{k} раз в \textit{n} независимых испытаниях, - в точности это количество сочетаний из \textit{n} по \textit{k}:

$$ C_{n}(k)={\frac  {n!}{k!\left(n-k\right)!}} $$.

В то же время, так как все испытания независимы и их исходы несовместимы (событие \textit{A} либо наступает, либо нет), то вероятность получения "удачной" комбинации в точности равна: $ p^{k}\cdot q^{n-k} $

Окончательно, для того чтобы найти вероятность того, что в \textit{n} независимых испытаниях событие \textit{A} наступит ровно \textit{k} раз, нужно сложить вероятности получения всех "удачных" комбинаций. Вероятности получения всех "удачных" комбинаций одинаковы и равны $ p^{k}\cdot q^{n-k}$, количество "удачных" комбинаций равно $ C_{n}(k) $, поэтому окончательно получаем:

 $$ P_{{k,n}}=C_{n}^{k}\cdot p^{k}\cdot q^{{n-k}}=C_{n}^{k}\cdot p^{k}\cdot (1-p)^{{n-k}}$$.

Последнее выражение есть не что иное, как Формула Бернулли. Полезно также заметить, что в силу полноты группы событий, будет справедливо:

$$ \sum _{{k=0}}^{n}(P_{{k,n}})=1 $$.

\end{document}